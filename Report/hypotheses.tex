\section{Hypotheses \& design choices}

The goal of this assignment was to create an XSD file to specify how a bookshop
should store information in XML files. This small report summarizes the
hypotheses and design choices we made.

First of all, the naming convention we chose is PascalCase for elements and
PascalCase for attributes in order to have consistent names.

The Root element is called BookShop. We also could have called it
Department or BookShopDepartment, but the first name does not give enough
information (department of what?), while the second one might be too verbose.

The \emph{ScientificProducts} element contains the \emph{Books} and the
\emph{Journals} elements. The \emph{Books} element contains \emph{Book} elements
and the \emph{Journals} element contains \emph{Journals} elements.
We could have stored both the \emph{Book} and \emph{Journal} elements directly
as children of \emph{ScientificProducts}, but by doing so, if we wanted to
retrieve a specific \emph{Book}, we would have to take the risk of unnecessarily
having to skim trough some \emph{Journal} elements.
The same reasoning applies to \emph{LeisureProducts}, \emph{Books} and
\emph{Periodicals}.
\begin{lstlisting}
<BookShop ...>
    <ScientificProducts>
        <s:Books>
            <s:Book>...</s:Book>
            <s:Book>...</s:Book>
            ...
        </s:Books>
        
        <s:Journals>
            <s:Journal>...</s:Journal>
            <s:Journal>...</s:Journal>
            ...
        </s:Journals>
    </ScientificProducts>
    ...
</BookShop>
\end{lstlisting}




% TODO diagram

% TODO refactoring done to avoid repetition
% TODO talk about groups being similar to composition in OOP and base to
% inheritance

% TODO split into namespaces
