\section{Hypotheses \& design choices}

\subsection{General}

First of all, the naming convention we chose is PascalCase for elements and
PascalCase for attributes in order to have consistent names.\\

The Root element is called BookShop. We also could have called it
Department or BookShopDepartment, but the first name does not give enough
information (department of what?), while the second one might be too verbose.\\

The \emph{ScientificProducts} element contains the \emph{Books} and the
\emph{Journals} elements. The \emph{Books} element contains \emph{Book} elements
and the \emph{Journals} element contains \emph{Journals} elements.
We could have stored both the \emph{Book} and \emph{Journal} elements directly
as children of \emph{ScientificProducts}, but by doing so, if we wanted to
retrieve a specific \emph{Book}, we would have to take the risk of unnecessarily
having to skim trough some \emph{Journal} elements.
The same reasoning applies to \emph{LeisureProducts}, \emph{Books} and
\emph{Periodicals}.
\begin{lstlisting}
<BookShop ...>
    <ScientificProducts>
        <s:Books>
            <s:Book>...</s:Book>
            <s:Book>...</s:Book>
            ...
        </s:Books>
        
        <s:Journals>
            <s:Journal>...</s:Journal>
            <s:Journal>...</s:Journal>
            ...
        </s:Journals>
    </ScientificProducts>
    ...
</BookShop>
\end{lstlisting}

\subsection{Elements \& attributes \small{
    \cite{cite:ibmGuidelines} \cite{cite:soAttrElem}
    \cite{cite:ukGovTalkGuidelines}
}}

There are no rules regarding what should be elements or attributes, the choice
is mostly arbitrary. However, we can follow some common guidelines.\\

First, if a type of information should appear multiple times (e.g. \emph{Author}
element in \emph{LBookType}), then it would be better to store it in a element.\\

Essential data, what the XML file is about, should be stored in elements, while
metadata should be stored in attributes (e.g. the \emph{PriceType} contains a
decimal value representing the price, and an attribute corresponding to the
\emph{currency}). Metadata could be, for example, data giving more information
about an element, information about how an
application should process the data (e.g. encoding), or an unique identifier.\\

The \emph{PriceType} complex type possesses a \emph{currency} attribute (e.g.
\verb+<price currency="EUR">10</price>+). We could have said that the
\emph{PriceType} should contain both the number and the currency in the its
value, but it would have been impractical if we had to do operations on the
number later. It would have required to parse the string to extract the number.
We also could have put the currency in an element instead of an attribute, but
it is more metadata than essential data.
With the way we handled the price, we could easily modify our schema to allow
multiple \emph{Price} elements for different currencies in \emph{Periodical}.

\begin{lstlisting}
<xsd:complexType name="PriceType">
    <xsd:simpleContent>
        <xsd:extension base="xsd:decimal">
            <xsd:attribute name="currency" type="CurrencyType"
                use="required" />
        </xsd:extension>
    </xsd:simpleContent>
</xsd:complexType>

<xsd:simpleType name="CurrencyType">
    <xsd:restriction base="xsd:string">
        <xsd:enumeration value="EUR" />
        <xsd:enumeration value="USD" />
    </xsd:restriction>
</xsd:simpleType>
\end{lstlisting}

\subsubsection{Restrictions}

Restrictions can be applied to types in order to check that some conditions
are met. That is the case with the \emph{ISBNType} and \emph{GenreType}.\\

In the first case, the ISBN in \emph{SBookType} is stored as an \emph{ISBNType}
which is a number on which restrictions have been defined. The first restriction
is a pattern checking if the number has 10 digits, while the second one checks
if the number has 13 digits. One of the two restriction must be validated in
order for the ISBN to be valid. An ISBN also possesses groups separated by
hyphens, but we do not store them neither do we check groups.
\begin{lstlisting}
<xsd:simpleType name="ISBNType">
    <xsd:restriction base="xsd:decimal">
        <xsd:pattern value="[0-9]{10}" />
        <xsd:pattern value="[0-9]{13}" />
    </xsd:restriction>
</xsd:simpleType>
\end{lstlisting}
\

In the second case, the assignment specific that the \emph{Genre} in
\emph{LBookType} should be restricted to the following values: thriller, horror,
sci/fi, romance and literature. In order to do that, we used XSD's
\emph{enumeration}.
\begin{lstlisting}
<xsd:simpleType name="GenreType">
    <xsd:restriction base="xsd:string">
        <xsd:enumeration value="thriller" />
        <xsd:enumeration value="horror" />
        <xsd:enumeration value="sci/fi" />
        <xsd:enumeration value="romance" />
        <xsd:enumeration value="literature" />
    </xsd:restriction>
</xsd:simpleType>
\end{lstlisting}

\subsection{Subclassing \& composition}

Like Object-Oriented Programming, XSD has subclassing and composition mechanisms.
Subclassing can be achieved trough the \emph{xsd:extension} element and its
\emph{base} attribute. Composition can be achieved with the \emph{xsd:group}
element. The problems are the same as with OOP, composition should be preferred
to inheritance if it can be applied. Problems such as needing to modify the
children if the base element is modified can arise with subclassing.
\newpage

As we can see in the diagram below, \emph{SBookType} and \emph{LBookType} are
subclasses from \emph{BaseBookType}. \emph{SBootType} and \emph{SJournalType}
also contain the \emph{AuthorEditorXORGroup} (composition).\\

\begin{figure}[h!]
    \includegraphics[width=0.6\textwidth]{images/subclassingCompositionDiag.png}
    \centering
    \caption{Relationship between (some) types}
\end{figure}

We could also have put the \emph{Title} and \emph{Publisher} in a base element
or group called \emph{BaseProduct}. However, it would only have been used by
\emph{BaseBookType} and \emph{LPeriodicalType} since \emph{SJournalType}'s
\emph{Publisher} element is optional. If only some of the products are
\emph{BaseProduct}s, then it cannot be considered a \emph{BaseProduct}.

\begin{framehint}
\textbf{N.B.} The diagram is not represented in a standard way, it takes some
elements from UML in order to try to describe our design choices more easily.
The boxes represent element types, which own other elements and attributes
represented by fields. The number of occurences is represented in the following
way: \emph{\{0..n\}} to specify that an element is optional and can occur multiple
times.
\end{framehint}

\newpage



% TODO split into namespaces

% TODO subsections
